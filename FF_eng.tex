\documentclass[helvetica,english,logo,notitle,totpages,utf8]{europecv2013}
\usepackage{graphicx}
\usepackage[a4paper,top=1.2cm,left=1.2cm,right=1.2cm,bottom=2.5cm]{geometry}
\usepackage[english]{babel}
\usepackage[T1]{fontenc}
\usepackage{relsize}
\usepackage{lipsum}
\usepackage{hyperref}
\graphicspath{{../pdf/}{images}}

\newcommand\CPP{C\nolinebreak[4]\hspace{-.05em}\raisebox{.4ex}{\relsize{-3}{\textbf{++}}}}
\newcommand{\tra}[1]{``#1''} %virgolette


\newenvironment{absolutelynopagebreak}
{\par\nobreak\vfil\penalty0\vfilneg
	\vtop\bgroup}
{\par\xdef\tpd{\the\prevdepth}\egroup
	\prevdepth=\tpd}

%[Tutti i campi del CV sono facoltativi. Rimuovere i campi vuoti.]
\ecvname{Francesco Farina}

\ecvaddress{UK: Flat 54, Iceland Wharf, Plough Way,  SE16 7AB, UK \newline \hspace*{16pt}IT: Via Nofilo 13, Pellezzano (SA), 84080, Italy}
\ecvtelephone{UK: +44 (0)7847725203}
\ecvemail{mail.farinafrancesco@gmail.com}
\ecvhomepage{\href{https://indiependente.dev}{indiependente.dev}}
\ecvlinkedin{\href{https://www.linkedin.com/in/francesco-farina-0579a6147}{linkedin.com/in/francesco-farina-0579a6147}}
\ecvgender{Male}
\ecvdateofbirth{17/10/1991}
\ecvnationality{Italian}

\ecvfootnote{Latest update: \today \\ I authorize the handling of my personal data pursuant to the Personal Data Protection Code Italian Leg. Dec. n. 196/2003.}
\ecvbeforepicture{\raggedleft}
\ecvpicture[width=2.5cm]{linkedin.jpeg}
\ecvafterpicture{\ecvspace{-20mm}}

\begin{document}
\selectlanguage{english}

\begin{europecv}
\ecvpersonalinfo[-2pt]

%\ecvposition{Posizione per la quale si concorre
%Posizione ricoperta
%Occupazione desiderata
%Titolo di studio per la quale si concorre}{Sostituire con posizione per la quale si concorre / posizione ricoperta / occupazione desiderata / titolo per il quale si concorre (eliminare le voci non rilevanti nella colonna di sinistra)}

\ecvsection{Work Experience}
%[Inserire separatamente le esperienze professionali svolte iniziando dalla più recente.]
% \par\vspace{3pt}
\ecvworkexperience{Jan 2021 - in progress}{Technical Lead}{River Island}{1 Curtain Place EC2A 3AN, London, UK}{\emph{RI Tech - MatTech Team}: Leading a new backend team in building a scalable customer backbone with the team by: driving the product architecture, promoting collaboration and engagement, integrating other internal systems and third parties, assuring code quality, controlling technical debt, ensuring healthy code reviews and safe releases.}
\ecvworkexperience{Apr 2018 - Dec 2020}{Microservices Engineer}{River Island}{1 Curtain Place EC2A 3AN, London, UK}{\emph{RI Tech - Orders \& Payments Team}: Engineering Golang microservices using TDD practices on the AWS cloud platform. Serverless (Lambda) and containerized (ECS) deployments depending on service requirements. Heavy use of Amazon Web Services including Lambda, S3, EC2, Kinesis, SQS, DynamoDB and Aurora RDS (MySQL and PostgreSQL). Infrastructure as code using HashiCorp’s Terraform. Concourse CI/CD pipelines.}
\ecvworkexperience{Apr 2017 - Apr 2018}{Technical Support Engineer}{ServiceNow}{1 Bridge Street, Staines-upon-Thames, UK}{\emph{Customer Support - Integrations}: email infrastructure, Single Sign-On, Data Import/Export, Web Services. Building tools for task automation in Go.}


\ecvsection{Education and Training}
%[Inserire separatamente i corsi frequentati iniziando da quelli più recenti.]
\ecverasmus[3pt]{Oct 2016 - Apr 2017}{Erasmus+ Internship}{ServiceNow, UK}{In-depth knowledge of the ServiceNow platform acquired through attendance of the System Administration, Scripting, Discovery, Application Creation classes, applying and enhancing it in internal projects in the Training department.}{Dr. Raffaele Manfellotto}


\ecveducation[3pt]{Dec 2013 - Sept 2016}{Master's Degree in Computer Science}{University of Salerno - Department of Computer Science}{Specialization in computer networks, parallel and concurrent computing, grid and cloud computing, distributed systems, data analysis, data integration, computational and artificial intelligence, security and cryptography, compilers, virtualization, advanced algorithms, social networks structure and robotics.}{EQF Level 7}{110/110 cum Laude}{\tra{A more efficient implementation of the subgraphs-world for the Glauber Dynamics in the Ising Model}, supervisor: Prof. V. Auletta}
\ecveducation[3pt]{Sept 2010 – Dec 2013}{Bachelor’s Degree in Computer Science}{University of Salerno - Department of Computer Science}{Programming languages, operating systems, algorithms, data structures, computer networks, software engineering, parallel and distributed programming, web development and database design.}{EQF Level 6}{110/110 cum Laude}{\tra{Aided-Design of agent-based simulations: the architecture of Agent Modeling Platform}, supervisor: Prof. V. Scarano}

%\ecvstdeducation[3pt]{09/2005 – 07/2010}{Diploma di Perito Tecnico Informatico}{Istituto Tecnico Industriale Statale \tra{Basilio Focaccia} (SA)}{Materie: informatica, sistemi, elettronica, calcolo statistico, matematica, inglese.}{V Livello QEQ}{95/100}

\ecvsection{Personal Skills}

\ecvmothertongue[3pt]{Italian}
\ecvlanguageheader
\ecvlastlanguage{English}{C1}{C1}{C1}{C1}{C1}


\ecvlanguagefooter[5pt]

\ecvitem[3pt]{Communication skills}
{I have good communication skills gained by participating in teams during my work and academic career.}

\ecvitem[3pt]{Organizational / Managerial skills}
{I worked in team made of 2 up to 8 people for academical and work projects. I have good job scheduling and problem solving competence, even when time is a critical factor.}

\ecvitem[3pt]{Professional skills}
{Thanks to the experience acquired in the last years, I have great problem solving skills, from a computer science related point of view. I can quickly analyze problems and pinpoint the methodologies needed for solving them, in order to provide the best feasible solution in the given schedule. I am a fast learner, therefore I can quickly learn new technologies and methodologies.}

% \ecvitem[3pt]{Technical skills}
% {Advanced knowledge of macOS, Unix-Based and Microsoft Windows operating systems. Great knowledge of Go, Python, Java and Javascript, good knowledge of C, PHP, HTML, CSS, XML, JSON, BASH Scripting, Matlab, Prolog, XPath, XQuery and \LaTeX. Advanced knowledge of Golang stdlib, Java SE platform, Docker. Good knowledge of the following libraries/frameworks: C Standard Lib, OpenMPI, Apache Hadoop, Java RMI, Java Swing, jQuery, AngularJS, Twitter Bootstrap, Express.js, Apache Axis2, Java FLEX/CUP, Numpy, Pandas, Scikit-learn, NetworkX, Matplotlib. Good knowledge of Assembly MIPS and x86-64. Competences of relational database MySQL, SQL and NoSQL database eXist. Basic knowledge of Prolog, Nvidia CUDA, Haskell. Advanced knowledge of Git version control. Basic knowledge of penetration testing methodologies and tools. Intermediate knowledge of blockchain theory.}


%\ecvitem[3pt]{Other skills}
%{My passion for music brought me to explore the music scene searching for new sounds to listen to with Hi-Fi equipment in order to get the best audio experience. I can play acoustic and electric guitar at a basic level. I have a basic knowledge of music production with Reaper DAW, Native Instruments Massive and Ableton Live 9 Suite.
%{\par\vspace{3pt}
%During my secondary school I obtained two english language certification: Trinity ISE 1 (B1 level in CEFR) and Trinity Grade 7 (B2 level in CEFR).}
%}
%\ecvsection{Additional Information}
%\ecvitem[10pt]{ServiceNow projects}{
%	\begin{itemize}
%		\item[\color{curious-blue}\tiny$\blacksquare$] I have worked on three applications for the training department coordinators integrating the Google Maps and Timezone API using UI Page, developing a tree-relationship structure for Location table, ACL design for lab instance images deployment and lab test results data integration between students' and instructor's instances.   
%	\end{itemize}
%}
\ecvitem[5pt]{Main Academic Projects}{
\begin{itemize}
	\setlength\itemsep{1pt}
	\item[\color{curious-blue}\tiny$\blacksquare$] Master Thesis subject: computing the Gibbs measure of the subgraphs-world dynamics for the Ising model on real world big datasets.
	\item[\color{curious-blue}\tiny$\blacksquare$] Car driver's condition tool [Python] that computes the Arousal level using a hypo-vigilance driver.
%	\item[\color{curious-blue}\tiny$\blacksquare$] Survey for the class of “Networks security” about Smart Grids and their security.
	\item[\color{curious-blue}\tiny$\blacksquare$] For the “Structures of the social networks” class, evaluated the influence of the most important nodes by implementing and applying centrality measures and one diffusion model to a real-world sample.
	\item[\color{curious-blue}\tiny$\blacksquare$] For the “Robotics” class, together with two colleagues, I built a gesture controlled vehicle using the Intel Galileo board as controller and the C++ language.
	\item[\color{curious-blue}\tiny$\blacksquare$] In a three member team, for the class “Data integration on web”, I developed an NBA players and teams data statistics web application, by gathering and integrating them from several websites using Node.js, Express and Angular.js.
	\item[\color{curious-blue}\tiny$\blacksquare$] During the class “Programming languages and compilers”, in a team, I developed the lexical, syntactic and semantic analysis modules for a compiler for the didactic language COOL.
	\item[\color{curious-blue}\tiny$\blacksquare$] I developed a scalable application, in a six members team, for the class “Advanced operating system”, based on MapReduce paradigm and the framework Apache Hadoop2, for the sequence alignment of genomics and proteomics. Also with some of them, I configured and maintained the forty nodes Hadoop cluster.
	% \item[\color{curious-blue}\tiny$\blacksquare$] For the class “Computer networks II”, in a three members team, I contributed to a Firefox extension called NoTrace by developing a graphic visualization of lost information while browsing, using the libraries Sigma.js and Twitter Bootstrap.
	%\item[\color{curious-blue}\tiny$\blacksquare$] Bachelor Thesis subject: I integrated, with another graduating colleague, the MASON library in a visual design system for agent-based simulations called Agent Modeling Platform, with automated generation of Java code, based on Eclipse, Java, Xpand, Xtend, EMF and PDT.
	\item[\color{curious-blue}\tiny$\blacksquare$] Before deep diving into Golang, I studied the Node.js platform and its most relevant modules. I published five modules on \url{http://npmjs.org}. The development was versioned by git, hosting the code on GitHub. I studied the open source licenses and learned its deploy process.
\end{itemize}
}



%\ecvitem[3pt]{Legal}{I authorize the handling of my personal data pursuant to the Personal Data Protection Code Italian Leg. Dec. n. 196/2003.}



%\ecvsection{Allegati}

%\ecvitem[10pt]{}{Sostituire con la lista di documenti allegati al CV. Esempio:
%\begin{itemize}
%\item copie delle lauree e qualifiche conseguite; 
%\item attestazione di servizio;
%\item attestazione del datore di lavoro.
%\end{itemize}}

\end{europecv}
\end{document} 
